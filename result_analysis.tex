We perform computation of LCS over 1000 pairs of random strings and the result shows that the gas fee before and after adopting our overpass protocol will be discussed. We use the same gas fee as Ethereum (0.09\$) for estimation. The following table shows the average gas per LCS test case. 
\begin{table}[H]
\begin{tabular}{|l|l|l|}
\hline
                      & Original   & Overpass  \\ \hline
\textbf{gas}          & 16,990,800 & 3,538,362 \\ \hline
\textbf{gas fee(Wei)} & 1.86 e+17  & 3.54 e+16 \\ \hline
\end{tabular}
\end{table}
As shown in the table, the gas fee is 16,990,800 before applying overpass. After optimization, it only costs about 3,538,362 which is an 80% reduction in the gas fee. This result indicates that our overpass protocol optimizes smart contracts by providing cheaper gas consumption for complex computation problems.  

The reduction in gas fees is the value our protocol can make. Part of it will be rewarded to miners as incentives to carry out computation/transaction property and motivate them for future execution. It is also beneficial to maintain a decentralized system. 

Our protocol results in win-win cooperation where users reduce costs and miners receive incentives. 
