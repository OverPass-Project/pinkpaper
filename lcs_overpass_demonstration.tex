% Author: Frank, Wenjun
The demonstration is done with Longest Common Sub-sequence(LCS)\cite{holmgren_et_al:LIPIcs.ICALP.2022.73} being the Kernel Question. The best search algorithm has a time complexity of O(m*n), where m, n are the length of the two sequences, but Verifying, if a given Sequence is indeed a common Subsequence of the two, takes only O(n), n being the shorter length of the two Sequences. For this Kernel Question, a miner must submit the length of the LCS and the LCS itself as proof for verification.

In the demonstration, an LCS Overpass instance is first deployed on the Ganache Test Network as shown in Figure \ref{TestNet}. To showcase the superiority of the OverPass Protocol, the deployed instance also includes a “compute” method that computes the LCS problem fully on EVM(Figure \ref{LCS}).
\begin{algorithm}[H]
    \captionof{figure}[Terminal1]{Terminal 1: Testnet} \label{TestNet}
    \begin{algorithmic}[1]
            \State > cd testnet 
            \State > sh start\_gananche\_testnet.sh 100 
    \end{algorithmic}
\end{algorithm}
\begin{algorithm}[H]
    \captionof{figure}[Terminal2]{Terminal 2: LCS} \label{LCS}
    \begin{algorithmic}[1]
            \State > python3 demo.py LCS
            \State  contract address: 0xe78A...
            \State  times\_to\_compute: > 20
    \end{algorithmic}
\end{algorithm}

\begin{algorithm}[H]
    \captionof{figure}[Terminal1]{Terminal 3: LCSOverPass} \label{LCSOverPass}
    \begin{algorithmic}[1]
            \State > python3 demo.py LCSOverPass
            \State  contract address: 0xe92A...
            \State  times\_to\_delegate: > 20
    \end{algorithmic}
\end{algorithm}
\begin{algorithm}[H]
    \captionof{figure}[Terminal1]{Terminal 3: LCSOverPass} \label{miner}
    \begin{algorithmic}[1]
            \State > python3 miner
            \State Available orders:
            \State 1. listen <contract\_address>
            \State 2. unlisten <contract\_address>
            \State 3. min\_incentive <min\_incentive>
            \State 4. maximum\_duration <maximum\_duration>
            \State 5. get\_incentive
            \State > listen 0xe92A...
            \State > get\_incentive
    \end{algorithmic}
\end{algorithm}
Then, a User instance is created and many deterministic random test cases of different lengths and alphabet sets are generated and fed into the LCS OverPass instance with both “delegateCompute” (Figure \ref{LCSOverPass})and “compute”(Figure \ref{LCS}). The average Gas cost for the test cases computed in two different ways is recorded. 

And, of course, a Miner instance(Figure \ref{miner}) is created and to listen to the LCS Overpass Contract address and will provide advice for any Tasks whose reward exceeds a fixed number. More detailed instruction introduction can be found in \cite{mu_nathanael_liu_woo_chen_harish_2022}.
