% modified from https://github.com/ethereum/yellowpaper/blob/master/Paper.tex
%\usepackage{tweaklist}
\usepackage{cancel}
\usepackage{xspace}
\usepackage{graphicx}
\usepackage{multicol}
\usepackage{subfig}
\usepackage{amsmath}
\usepackage{amssymb}
\usepackage[a4paper,width=170mm,top=18mm,bottom=22mm,includeheadfoot]{geometry}
\usepackage{booktabs}
\usepackage{array}
\usepackage{verbatim}
\usepackage{caption}
%\usepackage{natbib}
\usepackage{float}
\usepackage{pdflscape}
\usepackage{mathtools}
\usepackage[usenames,dvipsnames]{xcolor}
\usepackage{afterpage}
\usepackage{tikzpeople}
\usepackage{tikz}
\usepackage[bookmarks=true, unicode=true, pdftitle={Ethereum Yellow Paper: a formal specification of Ethereum, a programmable blockchain}, pdfauthor={Dr. Gavin Wood},pdfkeywords={Ethereum, Yellow Paper, blockchain, virtual machine, cryptography, decentralised, singleton, transaction, generalised},pdfborder={0 0 0.5 [1 3]}]{hyperref}
%,pagebackref=true

\usepackage{tabu} %requires array.
\usepackage[backend=bibtex,style=numeric,natbib=true]{biblatex}
\usepackage{float}
\addbibresource{Biblio}
%This should be the last package before \input{Version.tex}
\PassOptionsToPackage{hyphens}{url}\usepackage{hyperref}
% "hyperref loads the url package internally. Use \PassOptionsToPackage{hyphens}{url}\usepackage{hyperref} to pass the option to the url package when it is loaded by hyperref. This avoids any package option clashes." Source: <https://tex.stackexchange.com/questions/3033/forcing-linebreaks-in-url/3034#comment44478_3034>.
% Note also this: "If the \PassOptionsToPackage{hyphens}{url} approach does not work, maybe it's "because you're trying to load the url package with a specific option, but it's being loaded by one of your packages before that with a different set of options. Try loading the url package earlier than the package that requires it. If it's loaded by the document class, try using \RequirePackage[hyphens]{url} before the document class." Source: <https://tex.stackexchange.com/questions/3033/forcing-linebreaks-in-url/3034#comment555944_3034>.
% For more information on using the hyperref package, refer to e.g. https://en.wikibooks.org/w/index.php?title=LaTeX/Hyperlinks&stable=0#Hyperlink_and_Hypertarget.

\makeatletter
 \newcommand{\linkdest}[1]{\Hy@raisedlink{\hypertarget{#1}{}}}
\makeatother
\usepackage{seqsplit}

% For formatting
%\usepackage{underscore}
%\usepackage{lipsum} % to generate filler text for testing of document rendering
\usepackage[english]{babel}
\usepackage[autostyle]{csquotes}
\MakeOuterQuote{"}

\usepackage[final]{microtype} % https://tex.stackexchange.com/questions/75140/is-it-possible-to-make-latex-mark-overfull-boxes-in-the-output#comment382776_75142
\usepackage{plantuml}
%\input{Version.tex}
% Default rendering options
\usepackage{amsmath,amssymb}
\usepackage{algorithm}
\usepackage[noend]{algpseudocode}
\usepackage{caption}
\usepackage{parskip}
\usepackage{amsthm}
\definecolor{pagecolor}{rgb}{1,0.90,1}
\def\YellowPaperVersionNumber{unknown revision}
\IfFileExists{Options.tex}{\input{Options.tex}}

\newcommand{\hcancel}[1]{%
    \tikz[baseline=(tocancel.base)]{
        \node[inner sep=0pt,outer sep=0pt] (tocancel) {#1};
        \draw[black] (tocancel.south west) -- (tocancel.north east);
    }%
}%


\DeclarePairedDelimiter{\ceil}{\lceil}{\rceil}
\newcommand*\eg{e.g.\@\xspace}
\newcommand*\Eg{e.g.\@\xspace}
\newcommand*\ie{i.e.\@\xspace}
%\renewcommand{\itemhook}{\setlength{\topsep}{0pt}  \setlength{\itemsep}{0pt}\setlength{\leftmargin}{15pt}}

\usepackage{listings}
\usepackage{color}
\usepackage{blindtext}
\usepackage{hyperref}
\usepackage{float}
\usepackage{listings, xcolor}

\definecolor{dkgreen}{rgb}{0,0.6,0}
\definecolor{gray}{rgb}{0.5,0.5,0.5}
\definecolor{mauve}{rgb}{0.58,0,0.82}

\definecolor{littledarkerpink}{rgb}{1,.88,.97}

\lstdefinelanguage{Solidity}{
	keywords=[1]{anonymous, assembly, assert, balance, break, call, callcode, case, catch, class, constant, continue, constructor, contract, debugger, default, delegatecall, delete, do, else, emit, event, experimental, export, external, false, finally, for, function, gas, if, implements, import, in, indexed, instanceof, interface, internal, is, length, library, log0, log1, log2, log3, log4, memory, modifier, new, payable, pragma, private, protected, public, pure, push, require, return, returns, revert, selfdestruct, send, solidity, storage, struct, suicide, super, switch, then, this, throw, transfer, true, try, typeof, using, value, view, while, with, addmod, ecrecover, keccak256, mulmod, ripemd160, sha256, sha3}, % generic keywords including crypto operations
	keywordstyle=[1]\color{blue}\bfseries,
	keywords=[2]{address, bool, byte, bytes, bytes1, bytes2, bytes3, bytes4, bytes5, bytes6, bytes7, bytes8, bytes9, bytes10, bytes11, bytes12, bytes13, bytes14, bytes15, bytes16, bytes17, bytes18, bytes19, bytes20, bytes21, bytes22, bytes23, bytes24, bytes25, bytes26, bytes27, bytes28, bytes29, bytes30, bytes31, bytes32, enum, int, int8, int16, int24, int32, int40, int48, int56, int64, int72, int80, int88, int96, int104, int112, int120, int128, int136, int144, int152, int160, int168, int176, int184, int192, int200, int208, int216, int224, int232, int240, int248, int256, mapping, string, uint, uint8, uint16, uint24, uint32, uint40, uint48, uint56, uint64, uint72, uint80, uint88, uint96, uint104, uint112, uint120, uint128, uint136, uint144, uint152, uint160, uint168, uint176, uint184, uint192, uint200, uint208, uint216, uint224, uint232, uint240, uint248, uint256, var, void, ether, finney, szabo, wei, days, hours, minutes, seconds, weeks, years},	% types; money and time units
	keywordstyle=[2]\color{teal}\bfseries,
	keywords=[3]{block, blockhash, coinbase, difficulty, gaslimit, number, timestamp, msg, data, gas, sender, sig, value, now, tx, gasprice, origin},	% environment variables
	keywordstyle=[3]\color{violet}\bfseries,
	identifierstyle=\color{black},
	sensitive=false,
	comment=[l]{//},
	morecomment=[s]{/*}{*/},
	commentstyle=\color{gray}\ttfamily,
	stringstyle=\color{red}\ttfamily,
	morestring=[b]',
	morestring=[b]"
}

\lstnewenvironment{solidity}
{
\lstset{
	language=Solidity,
	backgroundcolor=\color{littledarkerpink},
	extendedchars=true,
	basicstyle=\footnotesize\ttfamily,
	showstringspaces=false,
	showspaces=false,
	numbers=left,
	numberstyle=\footnotesize,
	numbersep=9pt,
	tabsize=2,
	breaklines=true,
	showtabs=false,
	captionpos=b
}
}{}