% Author: Mu
Smart contracts introduced by Ethereum \cite{wood2014ethereum} provide trust computing. Once a smart contract has been engaged, it will complete exactly how it was coded and no parties can interfere or change the result. However, the computation is usually expensive and slow. There exists a set of problems where there is signs showing that verifying answers is cheaper than searching for the answer. Hence well-defined incentive mechanism can be applied to delegate searching computation to untrusted computational power (off-chain server) while the completeness and soundness of the computation are still guaranteed:
\begin{itemize}
    \item  A delegation task is posted on smart contract together with a trusted verification function.
    \item  Some untrusted machines would listen to the task event and provide answers/advice that could help smart contract solve the problem.
    \item With the hints from advisor, the client on-chain can compute the legitimate answer in a cheaper way.
    \item If the hints is illegitimate given by a dishonest advisor, the client will accept with negligible probability.
\end{itemize}

\subsection{Driving Factor}
The project is driven by the Pros and Cons own by traditional computer and consented decentralized computers as shown in the following table:
\begin{footnotesize}
\begin{table}[H]
\begin{tabular}{|c|c|c|}
\hline
Function             & Pros           & Cons            \\ \hline
Traditional Computer & fast and cheap & un-trusted      \\ \hline
Consented Computer   & trusted        & slow, expensive \\ \hline
\end{tabular}
\end{table}
\end{footnotesize}https://www.overleaf.com/project/636b3aaceb6b7d60b598c921
OverPass works as a standard to balance the Pros of these two and achieve cheaper and faster trustworthy computation.
\subsection{Theoretical Background} \label{theory}
% Author: Mu

The theory of computation is shaped by the Interactive Proof (IP) system\cite{GMR}, where a strong, possibly malicious, prover interact with a weak verifier, and at the end of the computation, the client can output an answer achieving completeness and soundness. There are signs that many problems has cheaper verification algorithm than search algorithm(e.g., NP-complete, sorting). This triggered a novel idea on trust-worthy computation that not all work should be done on the trusted slow "computer", or verifier, as long as the un-trusted computational power can provide proof for the answer to the verifier. This would expand the computational power of consented computers (e.g. EVM) tremendously while the trust of the computation is maintained. The class of problem we focus  are problems with doubly efficient IP scheme\cite{ip_muggles}\cite{goldreich_doubly_efficient_ip} and some P problems with cheap verification system than existing search algorithm.\\

\begin{tikzpicture}[every node/.style={minimum width=1.5cm}]
\node[alice] (alice) {Verifier $\widetilde{O}(|s|)$};
\node[bob,evil, right= 4cm of alice, mirrored] (bob) {Prover $O(poly(|s|))$};
\draw[->, thick] (alice) -- coordinate[midway] (aux) (bob);
\node[above= 0.3cm of alice] (input) {$s\in\{0,1\}^*$, Language $\mathcal{L}$};
\node[below= 0.5cm of aux] (etc) {...};
\node[above= 2.4cm of etc] (title) {\textbf{Doubly Efficient Interactive Proof}:};
\node[below= 0.7cm of alice] (alice_1) {};
\node[below= 0.7cm of bob] (bob_1) {};
\draw[<-,thick] (alice_1)--(bob_1);
\node[below= 0.9cm of alice] (alice_2) {$\langle P,V\rangle(\mathcal{L},s)\in \{0,1\}$};
\end{tikzpicture}
\begin{itemize}
    \item Completeness: $Pr[\langle P,V\rangle(\mathcal{L},s)=1|s\in \mathcal{L}]\geq 1-negl_1(|s|)$
    \item Soundness:  $s\notin \mathcal{L}:\forall P^*: Pr[\langle P^*,V\rangle(\mathcal{L},s)=1]\leq negl_2(|s|)$
\end{itemize}
An incentive mechanism is made such that miners have the incentive to mine by executing the protocol honestly and the verifier has the incentive to participate and save gas fees from achieving the same goal. An example of the incentive mechanism is as follows:
\begin{footnotesize}
\begin{table}[H]
\begin{tabular}{|c|c|c|}
\hline
client \textbackslash advisor & Honest                          & Cheat                \\ \hline
use OverPass                  & (a - $gas_{vrfy}$-i, i- $gas_vrfy$) & (0, - $gas_{vrfy}$)     \\ \hline
not user OverPass             & (a- $gas_{search}$, 0)             & (a - $gas_{search}$, 0) \\ \hline
\end{tabular}
\end{table}
\end{footnotesize}
\begin{itemize}
    \item a: the incentive got by user for getting the legitimate answer of the problem.
    \item i: the incentive paid by user to advisor.
    \item $gas_{vrfy}$: the gas fee for verifying an answer.
    \item $gas_{search}$: the gas fee for searching an answer.
\end{itemize}
Given that gas for searching is much higher than gas for verifier, and a wise incentive is set by the client, the (client use OverPass,Advisor be honest) is the Nash Equilibrium. Note that the mechanism would work under the assumption that the client is a rationale and at least one advisor is rationale and selfish, which is very robust.


\subsection{Previous Work} \label{ch:previous}
% Author:Joshua
There are currently products also serve on improving the price and speed of trustworthy computation, which are Solana and Arbitrum. First, Solana is a layer 1 blockchain that uses Proof of Stake (PoS) with Proof of History (PoH) to accelerate its average block validation times, which is around 10 times faster than Ethereum \cite{tyson_2022}. It also has a lower gas fee per transaction, which is around \$0,0000014 per transaction \cite{beincrypto_2022}. Second, Arbitrum is a layer 2 blockchain that is built on top of Ethereum. It uses Interactive Fraud Proofs, which is a type of Interactive Proofing. The basic idea of Interactive Fraud Proofing is that both the prover and verifier will try to check/bisect each other answers until they disagree to one or more things. This will, then, be run on the main Ethereum network. On the other hand, our project will be focusing on solving problems with cheap NP proof or doubly efficient interactive proofing, as shown in Section \ref{theory}.
